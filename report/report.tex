\documentclass{article}

\title{Benchmarking the Internet Computer}
\date{2023-02-13}
\author{Douglas Bouchet}

\begin{document}
\pagenumbering{gobble}
\maketitle
\newpage
\pagenumbering{arabic}

\tableofcontents

\newpage
\section{Abstract}
This paper presents a study on the use of blockchain technology as a platform for performing heavy computational tasks
such as machine learning. Through a series of tests, the paper attempts to identify the capabilities of blockchain for
such applications. The results of the research indicate that the size of transactions is limited to ten thousand values
and that there is a trade-off between redundancy and pace of execution. Furthermore, the length of the model has an
influence on the pace of the model. These results provide insights into the potential of blockchain for heavy
computational tasks and can be used to inform future research and development in this area.
\newpage
\section{Introduction}
The emergence of blockchain technology and its associated smart contracts has revolutionized the way we think about data
storage and computing. As a (student) researcher interested in the potential of blockchain technologies, I sought to explore the
ability of blockchain to provide a transparent and secure environment for performing heavy computational tasks, such
as machine learning. This paper aims to investigate if blockchain can indeed be used for such purposes.
In order to see if the blockchain can then be used to help with heavy computational tasks, we will take an already
existing protocol to perform this task, the Federated Learning. We will then see how it is possible to modify it by adding the blockchain.
We will detail the smart contracts used, as well as the interactions between the different agents involved in the
machine learning task. Finally we will test our new framework using diablo, a program that allows to submit transactions
to a blockchain, and to measure the performance of the latter, in response to these transactions. For more realism,
 the nodes constituting our blockchain will be emulated using machines made available by the AWS service

DO we add these ? -------
The expectations of some fundation such as Dfinity project
Can be questionable, as performaces of some BC can be quite limited
--------------
\section{Method}
\subsection{The need for comp. resources in ML}
- Powerfull models
- require lots data and computational power
\subsection{3 main goals of scaling up comp. power}
\subsection{Federated Learning, an existing solution to having computational resources}
- brief explanation of the protocol
\subsection{Advantages of using a blockchain to enhance this protocol}
slides which answer the 3 goals
\subsection{Including BC in federated learning}
- explain smart contract architecture
- explain model potential stealing
\subsection{Solving the model stealing pb}
- commit reveal protocol
\subsection{Dealing with SC limited memory}
- soem numbers for max var size
- idea: split the models
- new protocol explanation
- formula with hashes and chunk
- possibility to store model on an external sotrage
- total expected memory usage on the smart contract
\section{Learning a model using blockchain result and observations}
- small intro about aws + diablo
\subsection{First limitation, txs size}
- explain protocol for testing
- add graph
- add explanation
\subsection{The Redundancy/Pace tradeoff}
- term definition
- add graph
- add explanation
\subsection{Minimum pace depending on model length}
- Do this, as it could provide some idea about time needed to train a model
- Fix redundancy, why 8
- add graph
- add explanation
\subsection{Expected training time for some classical models}
- add tab
- add analysis
\section{Conclusion}
- extend what was said in the abstract
- opening to small ml model + transfert learning
\section{References}
- any ref to number used in the report
- link to github repo
\end{document}
